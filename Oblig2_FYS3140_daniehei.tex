\documentclass[a4paper,norsk, 10pt]{article}
\usepackage[utf8]{inputenc}
\usepackage{verbatim}
\usepackage{listings}
\usepackage{graphicx}
\usepackage[norsk]{babel}
\usepackage{a4wide}
\usepackage{color}
\usepackage{amsmath}
\usepackage{float}
\usepackage{amssymb}
\usepackage[dvips]{epsfig}
\usepackage[toc,page]{appendix}
\usepackage[T1]{fontenc}
\usepackage{cite} % [2,3,4] --> [2--4]
\usepackage{shadow}
\usepackage{hyperref}
\usepackage{titling}
\usepackage{marvosym }
\usepackage{subcaption}
\usepackage[noabbrev]{cleveref}
\usepackage{cite}
\usepackage{hyperref}


\setlength{\droptitle}{-10em}   % This is your set screw

\setcounter{tocdepth}{2}

\lstset{language=c++}
\lstset{alsolanguage=[90]Fortran}
\lstset{alsolanguage=Python}
\lstset{basicstyle=\small}
\lstset{backgroundcolor=\color{white}}
\lstset{frame=single}
\lstset{stringstyle=\ttfamily}
\lstset{keywordstyle=\color{red}\bfseries}
\lstset{commentstyle=\itshape\color{blue}}
\lstset{showspaces=false}
\lstset{showstringspaces=false}
\lstset{showtabs=false}
\lstset{breaklines}
\title{FYS3140 Oblig 2}
\author{Daniel Heinesen, daniehei}
\begin{document}
\maketitle

\section*{2.1)}
\subsection*{a)}
$$
\frac{-i + 2z}{2+iz} = \frac{-i + 2a + 2ib}{2 + ia - b} = \frac{2a + i(2b-1)}{(2-b) +ia}
$$

Vi ganger så telleren med den conjugerte til nevneren

$$
\frac{(2a + i(2b-1)) ((2-b) -ia)}{(2-b)^2 - a^2} = \frac{2a(2-b) - 2ia^2 + i(2b-a)(2-b) + a(2b-1)}{(2-b)^2 - a^2}
$$

Som vi sorterer de reelle og imaginære delene får vi:

$$
u(x,y) = \frac{2a(2-b) + a(2b-a)}{(2-b)^2 - a^2}
$$

$$
v(x,y) = \frac{(2b-a)(2-b) - 2a^2}{(2-b)^2 - a^2}
$$

\subsection*{b)}

$$
e^{iz} = e^{i(a+ib)} = e^{-b + ia} = e^{-b}e^{ia}
$$

Vi bruker så Eulers formel:

$$
e^{-b}e^{ia} = e^{-b}[\cos(a) + i\sin(a)]
$$

Dette gir oss at

$$
u(x,y) = e^{-b}\cos(a)
$$

$$
v(x,y) = e^{-b}\sin(a)
$$

\section*{2.2)}
$$
\frac{d}{dz}(f(z)g(z) = \lim_{\Delta z \rightarrow 0} \frac{f(z+\Delta z)g(z+\Delta z) - f(z)g(z)}{\Delta z}
$$

Vi tar så å plusser på $f(z+\Delta z)g(z)-f(z+\Delta z)g(z)$ i telleren:

$$
\lim_{\Delta z \rightarrow 0} \frac{f(z+\Delta z)g(z+\Delta z) - f(z)g(z) + f(z+\Delta z)g(z)-f(z+\Delta z)g(z)}{\Delta z}
$$

Vi ser nå at vi kan skrive dette som

$$
\lim_{\Delta z \rightarrow 0} \left( f(z+\Delta z) \frac{g(z+\Delta z) - g(z)}{\Delta z} + g(z) \frac{f(z+\Delta z) - f(z)}{\Delta z}\right)
$$

Når vi tar grensen ser vi at

$$
\lim_{\Delta z \rightarrow 0}  f(z+\Delta z) = f(z)
$$

Vi ender dermed opp med at

$$
\frac{d}{dz}(f(z)g(z) = f(z) \frac{dg(z)}{dt} + g(z) \frac{df(z)}{dt}
$$

\section*{2.3)}

Jeg kommer til å bruke en litt annen notasjon for den deriverte her:
$$
\frac{d}{dz} f(z_0) = \lim_{\Delta z \rightarrow 0} \frac{f(z) - f(z_0)}{z-z_0}
$$

Der $z = z_0 + \Delta z$, og $\Delta z = z -z_0$. Vi kan nå skrive det i polarkoordinater


$$
\frac{d}{dz} f(r_0e^{i\theta_0}) = \lim_{\Delta z \rightarrow 0} \frac{f(re^{i\theta}) - f(r_0e^{i\theta_0})}{z-z_0}
$$

I første omgang skal vi la $\theta$ være kostant, og tilnærme og punktet via $r$

$$
\frac{d}{dz} f(r_0e^{i\theta_0}) = \lim_{\Delta r \rightarrow 0} \frac{f(r_0e^{i\theta})-f(re^{i\theta_0})}{re^{i\theta_0}-r_0e^{i\theta_0}}
$$
$$
= e^{-i\theta_0} \lim_{\Delta r \rightarrow 0} \frac{f(r_0e^{i\theta})-f(re^{i\theta_0})}{r-r_0}
$$

$$
= e^{-i\theta_0} \lim_{\Delta r \rightarrow 0} \frac{u(r,\theta_0) + iv(r,\theta) - u(r_0,\theta_0) - iv(r_0,\theta_0)}{r-r_0}
$$

$$
= e^{-i\theta_0} \left[\frac{\partial u}{\partial r} + i\frac{\partial v}{\partial r}\right]
$$

Vi skal nå gjøre det samme, men holde $r$ konstant:

$$
\frac{d}{dz} f(r_0e^{i\theta_0}) = \lim_{\Delta \theta \rightarrow 0} \frac{f(r_0e^{i\theta_0})-f(r_0e^{i\theta})}{r_0e^{i\theta}-r_0e^{i\theta_0}}
$$
$$
= \frac{1}{r_0} \lim_{\Delta \theta \rightarrow 0} \frac{f(r_0e^{i\theta_0})-f(r_0e^{i\theta})}{e^{i\theta}-e^{i\theta_0}}
$$

Vi skal nå gjøre et lite triks(tatt fra \cite{triks})

$$
= \frac{1}{r_0} \lim_{\Delta \theta \rightarrow 0} \frac{f(r_0e^{i\theta_0})-f(r_0e^{i\theta})}{\theta - \theta_0}\frac{\theta - \theta_0}{e^{i\theta}-e^{i\theta_0}}
$$

Vi kan så se på det siste leddet:

$$
\lim_{\Delta \theta \rightarrow 0} \frac{e^{i\theta}-e^{i\theta_0}}{\theta - \theta_0} = \lim_{\Delta \theta \rightarrow 0} \frac{\cos\theta + i\sin\theta - \cos\theta_0 - i\sin\theta_0}{\theta - \theta_0}
$$

$$
= \lim_{\Delta \theta \rightarrow 0}\left(\frac{\cos\theta - \cos\theta_0}{\theta - \theta_0} + i\frac{\sin\theta - \sin\theta_0}{\theta - \theta_0}\right)
$$

$$
= -\sin\theta_0  + i\cos\theta_0 = ie^{i\theta_0}
$$

Vi får da at

$$
\frac{d}{dz} f(r_0e^{i\theta_0}) = \frac{-ie^{i\theta_0}}{r_0} \lim_{\Delta \theta \rightarrow 0} \frac{f(r_0e^{i\theta_0})-f(r_0e^{i\theta})}{\theta - \theta_0}
$$

$$
= \frac{-ie^{i\theta_0}}{r_0} \left[\frac{\partial u}{\partial \theta} + i\frac{\partial v}{\partial \theta}\right] = \frac{e^{i\theta_0}}{r_0} \left[\frac{\partial v}{\partial \theta} - i\frac{\partial u}{\partial \theta}\right]
$$

Siden den deriverte er unik vet vi at

$$
\frac{e^{i\theta_0}}{r_0} \left[\frac{\partial v}{\partial \theta} - i\frac{\partial u}{\partial \theta}\right] = e^{-i\theta_0} \left[\frac{\partial u}{\partial r} + i\frac{\partial v}{\partial r}\right]
$$

Setter vi de reelle dele like, og det samme for de imaginære dele:

$$
\frac{\partial u}{\partial r} = \frac{1}{r}\frac{\partial v}{\partial \theta}
$$

$$
\frac{\partial v}{\partial r} = -\frac{1}{r}\frac{\partial u}{\partial \theta}
$$


\section*{2.4)}
\subsection*{a)}

$$
u = \frac{y}{(1-x)^2 + y^2}
$$

For å sjekke om dette er en harmonisk funksjon må vi sjekke om $\nabla^2 u = 0$. Vi finner først at

$$
\frac{\partial u}{\partial x} = -\frac{2(x-1)y}{((1-x)^2 + y^2)^2}
$$
$$
\frac{\partial u}{\partial y} = \frac{(1-x)^2-y^2}{((1-x)^2 + y^2)^2} 
$$

så

$$
\frac{\partial^2 u}{\partial x^2} = \frac{6(1-x)^2y - 2y^3}{((1-x)^2 + y^2)^3}
$$
$$
\frac{\partial^2 u}{\partial y^2} = -\frac{6(1-x)^2y - 2y^3}{((1-x)^2 + y^2)^3} = -\frac{\partial^2 u}{\partial x^2}
$$


Vi ser da enkelt at

$$
\nabla^2 u = \frac{\partial^2 u}{\partial x^2} +\frac{\partial^2 u}{\partial y^2} =  0
$$

og $u$ er harmonisk.

\subsection*{b)}

Vi bruker Cauchy-Riemann for å finne $v$:

$$
\frac{\partial v}{\partial y} = \frac{\partial u}{\partial x} = -\frac{2(x-1)y}{((1-x)^2 + y^2)^2}
$$

For å finne et uttrykk for $v$ integrerer vi dette uttrykket med hensyn på y,og får da at:

$$
v(x,y) = \frac{x-1}{(1-x)^2 + y^2} + \phi (x)
$$

Vi bruker så at

$$
\frac{\partial v}{\partial x} = -\frac{\partial u}{\partial y} = \frac{y^2 + (1-x)^2}{((1-x)^2 + y^2)^2} 
$$

Integrerer vi dette får vi at

$$
v(x,y) = \frac{x-1}{(1-x)^2 + y^2} + \psi (y)
$$

Vi kan så slå disse sammen å får at

$$
v(x,y) = \frac{x-1}{(1-x)^2 + y^2} + C
$$

Hvor $C$ er en kostant.

Vi kan så skrive om $u+iv$ til en funksjon av $z$. Vi bruker da at $x = \frac{1}{2}(z+\bar{z})$ og $y = \frac{1}{2i}(z-\bar{z})$. Siden begge funksjonene har lik nevner kan vi starte med å finne denne som en funksjon av $z$

$$
(1-x)^2 + y^2 = 1 - 2x + x^2 + y^2 = 1 - z -\bar{z} + \frac{1}{4}(z+\bar{z}) - \frac{1}{4}(z-\bar{z}) 
$$
$$
1 - z -\bar{z} + \frac{1}{4} z + \frac{1}{4} \bar{z} +\frac{z\bar{z}}{2} - \frac{1}{4} z - \frac{1}{4} \bar{z} +\frac{z\bar{z}}{2} = 1 - z - \bar{z} - z\bar{z}
$$

Vi finner da at

$$
u = \frac{y}{(1-x)^2 + y^2} = \frac{1}{2i}\frac{z-\bar{z}}{1 - z - \bar{z} - z\bar{z}}
$$
og
$$
v = \frac{x-1}{(1-x)^2 + y^2} + C = \frac{1}{2}\frac{z+\bar{z} + 2}{1 - z - \bar{z} - z\bar{z}} + C
$$

Da blir 

$$
u + iv = \frac{1}{2i}\frac{z-\bar{z}}{1 - z - \bar{z} - z\bar{z}} - \frac{1}{i}\frac{1}{2}\frac{z+\bar{z} + 2}{1 - z - \bar{z} - z\bar{z}} + C 
$$

$$
= \frac{1}{i}\frac{\bar{z}-1}{1 - z - \bar{z} - z\bar{z}} + C = i \frac{1-\bar{z}}{(1-\bar{z})(1-z)} +C = \frac{i}{1-z} + C
$$

Så vi kan skrive

$$
f(z) = \frac{i}{1-z} + C
$$

\subsection*{c)}
Vi skal nå sjekke at $v$ er harmonisk:

$$
\frac{\partial v}{\partial x} = \frac{y^2 -(1-x)^2}{((1-x)^2 + y^2)^2}
$$
$$
\frac{\partial v}{\partial y} = \frac{-2(x-1)y}{((1-x)^2 + y^2)^2}
$$

Og 

$$
\frac{\partial^2 v}{\partial x^2} = \frac{2(x-1)^3 -6(x-1)y^2}{((1-x)^2 + y^2)^3}
$$
$$
\frac{\partial^2 v}{\partial x^2} = -\frac{2(x-1)^3 -6(x-1)y^2}{((1-x)^2 + y^2)^3} = -\frac{\partial^2 v}{\partial x^2}
$$

Og vi ser da at

$$
\nabla^2 v = \frac{\partial^2 v}{\partial x^2} +\frac{\partial^2 v}{\partial y^2} =  0
$$

og $v$ er harmonisk.

\section*{2.5)}
Om funksjonen ikke har noen singulæriteter innenfor konturen, $C$, vi integrerer over, bruker vi

\begin{equation}
\oint_C f(z) dz = 0
\label{noSing}
\end{equation}

Om det er en singulæritet der, må vi bruke Cauchy's integralteorem

\begin{equation}
\oint_C \frac{f(z)}{z-z_0} dz= 2\pi\cdot f(z_0)
\label{Sing}
\end{equation}
\subsection*{a)}

Vi kan se at for 

$$
\oint_C \frac{\sin z}{2z-\pi} dz
$$

er det en singulæritet ved $z_0 = \pi/2$, som er innenfor konturen(gitt som $|z| = 3$. Vi bruker derfor \ref{Sing}. Så

$$
\oint_C \frac{\sin z}{2z-\pi} dz = \sin(\frac{\pi}{2})\cdot2\pi i = 2\pi i
$$

\subsection*{b)}
Vi har samme integral, men konturen, $C$, er definert ved $|z| = 1$. Da ligger singulæriteten utenfor. Vi bruker da \ref{noSing}, og får at

$$
\oint_C \frac{\sin z}{2z-\pi} dz = 0
$$

\subsection*{c)}

For

$$
\oint_C \frac{\sin 2z}{6z-\pi} dz
$$

Ser vi at singulæriteten er ved $z_0 = \pi/6$. Dette er innenfor konturen $C$ gitt ved $|z| = 1$. Vi må da bruke \ref{Sing}, så

$$
\oint_C \frac{\sin 2z}{6z-\pi} dz = \sin(2\frac{\pi}{6})\cdot2\pi i = \sqrt{3}\pi i
$$

\subsection*{d)}

For 

$$
\oint_C \frac{e^{2z}}{z-\ln 2} dz
$$

er $z = \ln 2 \approx 0.693$ er en singulæritet. $C$ er et kvadrat med hjørnene i $\pm 2$ og $\pm 2i$. Vi siden $\ln 2 < 2$ så er det en singulæritet på innsiden av $C$. Vi bruker da \ref{Sing}. Så

$$
\oint_C \frac{e^{2z}}{z-\ln 2} dz = 2\pi i e^{2\ln2} =2\pi i (e^{\ln2})^2 = 2\pi i 2^2 = 8\pi i
$$



\bibliography{ref} 
\bibliographystyle{plain}



\end{document}


