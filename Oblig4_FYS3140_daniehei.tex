\documentclass[a4paper,norsk, 10pt]{article}
\usepackage[utf8]{inputenc}
\usepackage{verbatim}
\usepackage{listings}
\usepackage{graphicx}
\usepackage[norsk]{babel}
\usepackage{a4wide}
\usepackage{color}
\usepackage{amsmath}
\usepackage{float}
\usepackage{amssymb}
\usepackage[dvips]{epsfig}
\usepackage[toc,page]{appendix}
\usepackage[T1]{fontenc}
\usepackage{cite} % [2,3,4] --> [2--4]
\usepackage{shadow}
\usepackage{hyperref}
\usepackage{titling}
\usepackage{marvosym }
\usepackage{subcaption}
\usepackage[noabbrev]{cleveref}
\usepackage{cite}


\setlength{\droptitle}{-10em}   % This is your set screw

\setcounter{tocdepth}{2}

\lstset{language=c++}
\lstset{alsolanguage=[90]Fortran}
\lstset{alsolanguage=Python}
\lstset{basicstyle=\small}
\lstset{backgroundcolor=\color{white}}
\lstset{frame=single}
\lstset{stringstyle=\ttfamily}
\lstset{keywordstyle=\color{red}\bfseries}
\lstset{commentstyle=\itshape\color{blue}}
\lstset{showspaces=false}
\lstset{showstringspaces=false}
\lstset{showtabs=false}
\lstset{breaklines}
\title{FYS3140 Oblig 4}
\author{Daniel Heinesen, daniehei}
\begin{document}
\maketitle

\section*{1)}
\subsection*{a)}
$$
\frac{\sin z}{3z},\qquad z = 0
$$

We are going to start by writing a Taylor expansion of $\sin z$

\begin{equation}
\sin z = z - \frac{z^3}{3!} + \cdots
\label{eq:sin}
\end{equation}

Which gives

$$
\frac{\sin z}{3z} = \frac{1}{z}\left(z - \frac{z^3}{3!} + \cdots \right) = 1-\frac{z^2}{3!} + \cdots
$$

Since we don't have any $b_n$-term --and therefor no singularity in the series--, $z = 0$ is a removable singularity.

\subsection*{b)}
$$
\frac{\cos z}{z^4}, \qquad z=0
$$

We can now take Taylor of $\cos z$

$$
\cos z = 1 - \frac{x^2}{2!}+\frac{x^4}{4!}-\frac{x^6}{6!} + \cdots
$$

Which gives

$$
\frac{\cos z}{z^4} = \frac{1}{z^4}\left(1 - \frac{x^2}{2!}+\frac{x^4}{4!}-\frac{x^6}{6!} + \cdots\right) = \frac{1}{z^4}-\frac{1}{2z^2}+\frac{1}{4!}-\frac{z^2}{6!} + \cdots
$$


The highest negative power in this series is $2$, this means that this function has a pole of order $2$.

\subsection*{c)}
$$
\frac{z^3-1}{(z-1)^3},\qquad z = 1
$$

We can use partial fraction decomposition to get that

$$
\frac{z^3-1}{(z-1)^3} = \frac{3}{z-1} + \frac{3}{(z-1)^2} + 1
$$

This is the Laurent series. Since the highest negative power is $2$, this has a pole of order 2

\subsection*{d)}
$$
\frac{e^z}{z-1},\qquad z = 1
$$

We can do a Taylor expansion of $e^z$ around $z=1$

$$
\frac{e^z}{z-1} = e +e(z-1) + \frac{e(z-1)^2}{2!}+\cdots
$$

Which gives

$$
\frac{e^z}{z-1} = \frac{1}{z-1}\left(e +e(z-1) + \frac{e(z-1)^2}{2!}+\cdots\right) = \frac{e}{z-1} + e + \frac{(z-1)e}{2!} + \cdots
$$
The greatest negative power here is 1, so this has a pole of order 1(simple point).

\subsection*{e)}
$$
\frac{e^z - 1- z}{z^3},\qquad z = 0
$$

We can use partial fraction decomposition to find

$$
\frac{e^z - 1- z}{z^3} = \frac{e^z}{z^3} - \frac{1}{z^3} - \frac{1}{z^2}
$$

We can use Taylor on $e^z$ which gives us

$$
\frac{e^z - 1- z}{z^3} = - \frac{1}{z^3} - \frac{1}{z^2} + \frac{1}{z^3}\left(1+z+\frac{z^2}{2!}+\frac{z^3}{3!}+\frac{z^4}{4!} + \cdots\right)
$$

$$
= - \frac{1}{z^3} - \frac{1}{z^2} + \frac{1}{z^3} + \frac{1}{z^2} + \frac{1}{2z} + \frac{1}{3!} + \frac{z}{4!} + \cdots
$$

$$
=\frac{1}{2z} + \frac{1}{3!} + \frac{z}{4!} + \cdots
$$

Which shows us that the function has a pole of order 1 (simple point).
\newpage
\section*{2)}
\subsection*{14.6.1)}

$$
\frac{1}{z(z+1)}, \qquad z = 0
$$

$z = 0$ for $\frac{1}{z+1}$ is not a singularity, so we are going to look at the Taylor expression for this expression

$$
\frac{1}{z+1} = 1-z+z^2-z^3 + \cdots
$$

So

$$
\frac{1}{z(z+1)} = \frac{1}{z}\left(1-z+z^2-z^3 + \cdots\right) = \frac{1}{z} - 1 +z -z^2 +\cdots
$$
The residue is the coefficient of the term $z^{-1}$. So the residue here is $\underline{\underline{1}}$



\subsection*{14.6.3)}
$$
\frac{\sin z}{z^4}, \qquad z = 0
$$

We have the Taylor expansion of $\sin z$ from \ref{eq:sin}. Thus

$$
\frac{\sin z}{z^4} = \frac{1}{z^4}\left(z - \frac{z^3}{3!} + \frac{z^5}{5!} + \cdots\right) = \frac{1}{z^3} - \frac{1}{6z} + \frac{z}{5!} + \cdots
$$

The residue is the coefficient of the term $z^{-1}$. So the residue here is $\underline{\underline{-\frac{1}{6}}}$

\subsection*{14.6.9)}
$$
\frac{1}{z^2 - 5z +6}, \qquad	 z=2
$$

We are going to start by doing a partial fraction decomposition

$$
\frac{1}{z^2 - 5z +6} = -\frac{1}{z-2} + \frac{1}{z-3}
$$

The first term is as it should in the Laurent series. We then have to expand $\frac{1}{z-3}$ around $z = 2$. We are going to use that

$$
\frac{1}{1-z} = \sum_{n = 0}^{\infty}z^n
$$

so

$$
\frac{1}{z-3} = -\frac{1}{1-(z-2)} = -\sum_{n = 0}^{\infty}(z-2)^n
$$

So the full function

$$
\frac{1}{z^2 - 5z +6} = -\frac{1}{z-2} -\sum_{n = 0}^{\infty}(z-2)^n
$$

The residue is the coefficient of the term $(z-2)^{-1}$, so the residue is $\underline{\underline{-1}}$

\end{document}


